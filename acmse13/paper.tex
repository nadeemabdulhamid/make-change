\documentclass{acm_proc_article-sp}

\title{Formal Verification of Change Making Algorithms}
\subtitle{(Poster Abstract)}
\numberofauthors{2}
\author{
\alignauthor
Nadeem Abdul Hamid\\
  \email{nadeem@acm.org}
\alignauthor
Brook Bowers\\
  \email{brook.bowers@vikings.berry.edu}
\and
  \affaddr{Berry College} \\
   \affaddr{Department of Mathematics and Computer Science}\\
  \affaddr{Mount Berry, GA 30149}\\
}

\begin{document}

\maketitle

\begin{abstract}
blah blah
\end{abstract}

\category{D.2.4}{Software/Program Verification}{Correctness proofs}
\category{F.3.1}{Specifying and Verifying and Reasoning about Programs}{Mechanical verification}

\terms{Verification}

\keywords{change making, formal methods, Coq proof assistant} % NOT required for Proceedings

\section{Introduction}

The \emph{change making problem} poses the task of determining how to represent a given value using the fewest coins from a given set of denominations. Formally, given a value $k$ and a finite set of positive integers, $c_1 > c_2 > \ldots > c_n$ (called a \emph{coin system}), where $c_n = 1$, we wish to find a sequence of nonnegative integer coefficients, $x_i,\ 1 \le i \le n$ (called a \emph{representation} of $k$) such that the size of the representation,
\[ \sum_{i=1}^{n}x_i \]
is minimized, subject to,
\[ k = \sum_{i=1}^{n}{x_i c_i}\ . \]
Change making is a variation of the general knapsack problem in which the number of items (coins) available is unbounded and the goal is to minimize the total number of items selected to fit in the ``knapsack'' (with capacity represented by the target value). In the general case, determining the optimal solution to the change making problem is \emph{NP}-hard~\cite{martello90}. Nonetheless, for particular coin systems (including almost all coin systems in use around the world), the problem can be efficiently solved using a greedy algorithm. 
%For instances where the greedy algorithm does not produce an optimal solution, a dynamic programming algorithm may be used instead (which always produces an optimal solution, but has higher computational complexity).


\section{Change Making Algorithms}

A recursive algorithm to compute $M(k)$, the size of the optimal representation for a value $k$, can be derived based on the recurrence:
\[
M(k) = \left\{ 
	\begin{array}{lr}
	0 & \textit{if}~k=0\\
	\textrm{min}_{i:c_i \le k}\left\{1 + M(k - c_i)\right\}   & \textit{if}~k>0
	\end{array}\right.
\]
In practice, the implementation of this definition must be memorized to avoid repeated subcomputations. Since the change making problem exhibits an optimal substructure property, the definition above may be used to implement a dynamic programming solution. These algorithms run in time $\mathcal{O}(nk)$ with space requirement in $\mathcal{O}(n)$. 


The greedy algorithm for the change making problem proceeds by repeatedly choosing to include the largest coin less than or equal to the amount remaining at a given stage of the process. The time complexity of this algorithm is $\mathcal{O}(n)$ - linear in the number of coins.  As noted, for some coin systems the greedy algorithm always produces an optimal representation for any given value - such coin systems are called \emph{canonical}~\cite{martello90}. However, it is easy to construct a coin system for which the greedy algorithm fails to produce an optimal representation for a particular value. An interesting question, therefore, is whether there is a way to determine if an arbitrary coin system is canonical. Since there are infinitely many values of $k$ to test for a given coin system, it is not immediately evident that a finite process exists to do so. A progression of results~\cite{chang70,kozen94,pearson94} has shown that, in fact, it is possible to determine canonicity (ultimately in time polynomial to the number of coins in the system) by examining a finite number of possible counterexamples. 


\section{Formalization}

Our goal in this project has been to develop formally verified implementations of algorithms for the change making problem. In the sections that follow, we describe our programming and reasoning environment and outline the stages of our formalization. 

\subsection{Coq Proof Assistant}

Our formal developments are  carried out using the Coq proof assistant~\cite{coq12}. The Coq system is based on 
the Calculus of Inductive Constructions (CIC)~\cite{paulin93}, a higher-order typed lambda calculus with dependent types and inductive definitions. It provides a uniform framework for use  both as a functional programming language and a higher-order logic for reasoning. In order for the logical system to remain consistent, certain constraints are imposed on the structure of data  and functions that can be defined in the language. For example, only terminating recursion can be expressed and all functions must be total. The ramifications of this from a programming perspective are that the language is not Turing-complete; one cannot write a non-terminating function in it. Nonetheless, since algorithms, by definition, must terminate, it is possible (in theory at least) to implement many interesting algorithms in Coq and prove properties about their behavior. 

\subsection{Implementation}

Our first task has been to implement the basic algorithms in Coq. Not only must these be expressed in pure functional style, any recursively defined function must be shown to terminate before it passes the Coq type system. In most cases, we strive to write functions using structural recursion on the input data (usually a list or a natural number), by virtue of which termination is easily proven. In some cases, it may be necessary to define some measure on the size of inputs to a function and show that the measure "decreases" with each recursive call. At the time of writing, a reasonably efficient formulation of a dynamic programming implementation is unclear due to the lack of imperative data structures in the language (e.g. arrays).

We also implement the ideas from~\cite{pearson94} in the form of an algorithm that generates and checks a set of counterexamples for a given coin system to determine if it is canonical - i.e. if the greedy algorithm will always produce the minimal representation for the system. The validity of the algorithm for finding a counterexample is non-obvious and non-trivial. This makes it even harder to be certain that a concrete implementation of the algorithm (such as ours) is actually correct and would always find a counterexample if it existed. This makes it a particularly good example of a situation where formal verification would be of great value. 

\subsection{Proofs}

The next step, having implemented the direct and greedy algorithms for making change along with the algorithm for determining canonicity, is to formal establish properties of our implementation. There are several general properties of interest to establish for each algorithm: termination, correctness, optimality, and efficiency. The first is achieved by virtue of the Coq language and type system: as our recursive definitions pass the basic type checker, termination is guaranteed. Our current focus is on proving correctness and optimality. Towards this end, we first establish formal definitions of each of these properties in the context of the change making problem. As anticipated, the effort required to formally prove the correctness of the implementations  is  an order of magnitude greater than the process of implementing the algorithms in the first place. 

To address optimality, we again follow~\cite{pearson94}, essentially formalizing the definitions of minimality and canonicity given and then formalizing the proof of correctness of the counterexample-generation algorithm. We then have a formal guarantee, for arbitrary canonical coin systems, that the implementation of the greedy algorithm always produces the optimal solution. On the other hand, for a non-canonical system, we are guaranteed that a counterexample will be generated, in which case a different algorithm (with guaranteed optimality, but higher computational complexity) can be used instead of the greedy one.

Establishing the fourth property of our implementations - efficiency - remains as possible future work. We are not aware of very many formalizations of computational complexity but it would seem to be as useful to have formal guarantees of  implementations' efficiency as well as their correctness.


\section{Conclusion}

We present work in progress towards formally verified implementations of various algorithms related to the change making problem. The problem provides a good context for applying formal techniques, as there are several critical, yet subtle and non-intuitive, properties upon which correctness and optimality depend, especially of the widely-used greedy algorithm.



\bibliographystyle{abbrv}
\bibliography{paper}  % the name of the Bibliography in this case


\end{document}